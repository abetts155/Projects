\documentclass{article}
\usepackage{graphicx}
\usepackage{listings}

\lstset{breaklines=true,basicstyle=\scriptsize\tt}
\newcommand{\code}[1]{\lstset{basicstyle=\footnotesize\tt}\lstinline£#1£\lstset{basicstyle=\scriptsize\tt}}

\begin{document}

\title{Using GPGPU-Sim to do Cost Modelling with Hybrid WCET Analysis}
\author{Adam Betts}

\maketitle

\newcommand{\gpusim}{GPGPU-sim }

\begin{abstract}
  This documents changes made to \gpusim to do cost modelling for the CARP project.
  It is a compendium of notes that also includes how to run it.
\end{abstract}

\section{Introduction}

Important notes follow:

\begin{itemize}
  \item I have only changed v3.x of the simulator.
\end{itemize}

\subsection{Simulator Source Files}

For our research, we need both the CFGs of each function, a call graph of the entire kernel,
and a timestamped trace of execution. \gpusim provides the possibility to extract all this
information, but only with some modifications to the source code. Here I explain which files
and which functions were touched. 

\textbf{cuda-sim.cc::function\_info::ptx\_assemble()}: Called before simulation starts to reconstruct 
the CFG of the kernel. In order to see the basic blocks (including PTX instructions) and the links
between them, we have to add:

\begin{lstlisting}
  print_basic_blocks();
  print_basic_block_links();
\end{lstlisting}

\textbf{shader.cc::LooseRoundRobbinScheduler::cycle()} Simulates a cycle on a multiprocessor by
deciding which instruction in the queue to issue. When the instruction is issued to the multiprocessor,
it means that particular warp is in flight. To produce a trace of when this event occurs we have to add:

\begin{lstlisting}
  if(warp_inst_issued) {
    printf("Issued: PC = 0x\%04x, Warp = \%d, SM = \%d, cycle = \%llu\n", pI->pc, warp_id, m_shader->m_sid, gpu_sim_cycle);
  }
\end{lstlisting}       

\textbf{ptx_ir.cc::function_info::connect_basic_blocks()} There is a bug in CFG generation when the last 
instruction of a basic block is a predicated return operation. In the original source code we have:

\begin{lstlisting}
  unsigned next_addr = pI->get_m_instr_mem_index() + 1;
\end{lstlisting}

which tries to find the address of the next instruction in the sequence. That is wrong as we have to 
add the instruction width instead as follows:
\begin{lstlisting}
  unsigned next_addr = pI->get_m_instr_mem_index() + pI->inst_size();
\end{lstlisting}

\subsection{Running}

Running \gpusim is a bit awkward. It consists of several steps:

\begin{itemize}
  \item Set environment variable CUDA\_INSTALL\_PATH to point to the base directory of your CUDA installation.
   Since I am modifying v3.x and running the Fermi-based simulator (explained below), it is required
   that you have CUDA v4.x or greater installed and that CUDA\_INSTALL\_PATH points to that installation. 

  \item Copy \textbf{all} the files from v3.x/configs/arch/ to the directory where the CUDA binary lives. 

  \item Run `source v3.x/setup\_environment' in the directory where the CUDA binary lives. At this point, in principle,
  everything is good to go.

  \item Run the CUDA binary as normal, i.e. './a.out'. If successful, you will see a splurge of output as \gpusim dumps
  various information related to that specific run.
\end{itemize}

\end{document}
